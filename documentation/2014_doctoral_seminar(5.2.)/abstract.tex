%%
%% Example
%%

\documentclass[a4paper,11pt]{article}

\usepackage[cp1250]{inputenc}
%\usepackage[czech]{babel}          %Windows
\usepackage{czech}

\begin{document}


{
\begin{center}
{\bf\MakeUppercase{Faculty of Mechatronics, Informatics and Interdisciplinary Studies \newline Technical University of Liberec}} \newline
Pavel Exner \newline
Partition of unity methods for approximation of point water sources in porous media
\end{center}
}

People often consider in their models of flow in porous media very large areas which can contain various phenomena of a very small scale
compared with the size of the areas. These can be some disruptions of the porous media, e.g. cracks and wells, or material inhomogeneities
which cause large gradients in pressure head and velocity or even their discontinuities.

Using the standard Finite Element Method (FEM) with linear finite elements we are unable to properly approximate the quantities in the vicinity of these 
disturbances, unless we introduce cells of the same scale in the mesh. This leads to very fine meshes and increases computational costs.

We use Partition of Unity Methods (PUM) to overcome this problem and demonstrate it on a steady quasi-three-dimensional model of multi-aquifer 
system containing hydro-geological wells which cause singularities in the solution. The corrected eXtended FEM (XFEM) 
and Stable Generalized FEM (SGFEM) will be presented.


\end{document}
