%%
%% Example
%%

%% Select document class - DO NOT CHANGE
\documentclass{sna}

%% Select any of standard LaTeX2e packages, e.g.
%  \usepackage{czech}
%  \usepackage{epsf}
%  \usepackage{graphicx}
\usepackage{amsmath}
%  \usepackage{amsfonts}
\usepackage{hyperref}

\begin{document}

%% Specify the paper: title, author(s), home institution and city
\info{Partition of unity methods for approximation of point sources in porous media}
     {P. Exner}
     {Faculty of Mechatronics, Informatics and Interdisciplinary Studies \\ Technical University of Liberec}

% Introduction
%\textbf{Keywords: } multi-aquifer system, water flow, PUM, XFEM, SGFEM
%\vspace{5pt}

People often consider in their models of flow in porous media very large areas which can contain various phenomena of very small scale
compared with the size of the areas. These can be some disruptions of the porous media, e.g. cracks and wells, or material inhomogeneities
which cause large gradients in pressure head and velocity or even their discontinuities.

Using the standard FEM (Finite Element Method) we are unable to properly approximate the quantities in the vicinity of these 
disturbances, unless we introduce cells of the same scale in the mesh. This leads to very fine meshes which highly increase computational costs.
We use XFEM method to overcome this problem and demonstrate it on a steady quasi-three-dimensional model of multi-aquifer system containing 
hydro-geological wells which cause singularities in solution. We follow the work~\cite{gracie} of R. Gracie and J.~R. Craig who have already used the XFEM on a~similar model.

We consider steady flow in a system of aquifers (2D layers of given thickness) which are separated by impermeable layers. The distribution of pressure head in each aquifer is described
by Poisson equation. The communication between aquifers is possible only through wells which can be seen as 1D problems. The transfer between wells and aquifers can be treated in two ways --
as a~balance over the boundary of the wells or as flow sources in the area of aquifer. We compared both approaches.


In the theoretical part of our work we derived the weak formulation of the problem and we proved the existence and the uniqueness 
of the weak solution according to Lax-Milgram lemma.

We implemented both the XFEM (eXtended Finite Element Method) and h-adaptive FEM with linear finite elements.
Measured XFEM convergence rate $O(h^{1.7})$ and FEM convergence rate $O(h^{0.4})$ are in good agreement with results of R. Gracie and J.~R. Craig in~\cite{gracie}.

Next, we used the SGFEM method (Stable Generalized FEM, introduced by U. Banerjee and I. Babu\v ska in~\cite{sgfem}) to solve the same model. This method improves
numerical properties of the linear system. We did also some time profiling, observed number of degrees according to size of enriched area and investigated 
numerical properties to show the benefits and also disadvantages of XFEM and SGFEM methods. 

Current work is aimed at using XFEM/SGFEM in mixed method to approximate both pressure head and velocity. There are several aspects to be solved -- choice
of enrichment functions and finite elements, combination of the methods together and also building of theoretical background.


%\cite{dip}

% V modelech proudění podzemních vod jsou často zahrnuty rozsáhlé oblasti, vůči kterým se jeví zdroje (např. hydrogeologické vrty) jako bodové. 
% Často používaná metoda konečných prvků nedokáže dostatečně přesně aproximovat tlak a~proudové pole v okolí těchto zdrojů. Naším cílem je implementace 
% metody XFEM (rozšířené metody konečných prvků) pro výpočet takových modelů a~porovnání s~metodou lineárních konečných prvků. Vycházíme z~článku~\cite{gracie}, 
% jehož výsledky se snažíme rekonstruovat vlastní implementací a následně obohatit.

% Naším cílem je implementovat metodu rozšířených konečných prvků (XFEM), pomocí které obohatíme prostor bázových funkcí o funkce, 
% které lépe vystihnou lokální charakter řešení v okolí vrtů. Pro porovnání implementujeme metodu lineárních konečných prvků (FEM) s adaptivně zjemňovanou sítí.
% Implementaci provedeme v~jazyce C++ pomocí knihovny Deal II (verze 7.2, viz~\cite{deal}). Tato knihovna zatím metodu XFEM nijak nepodporuje.




% \section{Model description}
% 
% The rock is often described as a system of several layers whose properties can differ one from another. 
% We can then simplify the problem of underground flow to problem of flow in layers with higher hydrological conductivity (aquifers) 
% which are separated by layers with smaller hydrological conductivity (aquitards). We will consider that the flow between the layers is 
% possible only through hydro-geological wells and that the aquitards are absolutely impermeable. Also the outer boundary of the aquifers and
% aquitards is considered impermeable so we can prescribe the homogeneous Neumann boundary condition.
% 
% % Na hranici zvodně zavádíme homogenní Neumannovu nebo Dirichletovu okrajovou podmínku. Vstupními daty jsou pak fyzikální konstanty, 
% % tlaky v ústích vrtů  a okrajové podmínky.
% 
% We describe the flow in each aquifer by Poisson equation which form the system of equations
% %
% % Proudění podzemní vody v hornině lze často zjednodušit na model proudění ve vrstvách (zvodních), neboť i hornina je tvořena vrstvami. 
% % Předpokládáme, že jednotlivé zvodně jsou od sebe odděleny nepropustnou vrstvou a komunikují mezi sebou pouze prostřednictvím vrtů. 
% % Zvodně budeme modelovat jako 2D plochy. Matematicky popíšeme takový model soustavou rovnic
% %
% \begin{eqnarray} 
% -T^m\Delta{}h^m&=&f^m  \qquad \textrm{na } \Theta^m, \label{eqn:poisson} \\
% %
% \int_{\partial{}B_w^m}\sigma_w^m \left(h^m - H_w^m\right) \, \mathrm{d}\mathbf{x} &=& c_w^{m+1}\left( H^m_w-H_w^{m+1}\right) - c^m_w\left( H^{m-1}_w-H^m_w \right),
% \label{eqn:well_eqs_test} \\
% && \forall m=1,\dots,M \;\textrm{ and } \forall w=1,\dots,W. \nonumber
% \end{eqnarray}
% %
% The indexes $m$ a $w$ mark the relation of quantities to $m$-th aquifer and to $w$-th well. Equation (\ref{eqn:poisson}) is derived from the Darcy law and
% the continuity equation. In equation (\ref{eqn:poisson}) $T^m\, [\textrm{m}^2\textrm{s}^{-1}]$ denotes transmisivity, $h^m\, [\textrm{m}]$ is pressure head 
% and $f^m\, [\textrm{m}\textrm{s}^{-1}]$ is the source density in $m$-th aquifer.
% The equation (\ref{eqn:well_eqs_test}) describes the balance between the inflow and the outflow in and out of the well on right hand side and the flow through
% the well boundary on the left hand side. In (\ref{eqn:well_eqs_test}) $\sigma^m_w\, [\textrm{m}\textrm{s}^{-1}]$ denotes the permeability coefficient between
% $w$-th well and $m$-th aquifer, $H_w^m$ stands for the pressure head in the well $w$ at the level of $m$-th aquifer, $c^m_w\, [\textrm{m}^2\textrm{s}^{-1}]$ 
% permeability of the well $w$ between aquifers and finally $\partial{}B^m_w$ is the boundary of the well.
% 
% 
% % Indexy $m$ a $w$ v~rovnicích označují vztah veličin k~$m$-té zvodni a k~$w$-tému vrtu. Rovnice (\ref{eqn:poisson}) je odvozena z~Darcyho zákona 
% % a~z~rovnice kontinuity pro nestlačitelnou tekutinu, za předpokladu izotropního prostředí. V (\ref{eqn:poisson}) 
% % značí $T^m\, [\textrm{m}^2\textrm{s}^{-1}]$ transmisivitu, $h^m\, [\textrm{m}]$ tlakovou výšku a~$f^m\, [\textrm{m}\textrm{s}^{-1}]$ hustotu zdrojů v~$m$-té zvodni.
% % Rovnice (\ref{eqn:well_eqs_test}) vyjadřuje rovnováhu mezi rozdílem přítoku a odtoku ve vrtu na pravé straně a tokem přes hranici vrtu 
% % do zvodně na levé straně rovnice. V~(\ref{eqn:well_eqs_test}) značí $\sigma^m_w\, [\textrm{m}\textrm{s}^{-1}]$ koeficient propustnosti 
% % mezi $w$-tým vrtem a~$m$-tou zvodní, $H_w^m$ tlakovou výšku ve vrtu $w$ na úrovni $m$-té zvodně, $c^m_w\, [\textrm{m}^2\textrm{s}^{-1}]$ propustnost 
% % vrtu mezi zvodněmi a~$\partial{}B^m_w$ hranici vrtu.
% 
% We can take a look at the boundary flow in (\ref{eqn:well_eqs_test}) from two different sides. The term resembles the Robin boundary prescription or it can 
% represent flow source in the area of the well and aquifer cross-section (the units of $\sigma^m_w$ then changes). In the first case a boundary integral will appear in 
% the weak formulation of (\ref{eqn:poisson}), in the second case a surface integral will appear in the source in (\ref{eqn:poisson})
% Both variants were tested in the work with nearly identical results, only the second approach simplifies the implementation and slightly speeds up the assembly.
% 
% 
% % Na vrty se dá nahlížet dvěma způsoby -- Newtonova okrajová podmínka na hranici vrtu, viz (\ref{eqn:well_eqs_test}), nebo zdroj v ploše 
% % průniku vrtu a zvodně (rovnice odvozeny v~diplomové práci). Obě varianty jsme otestovali s tím, že vrty brané jako plošné zdroje zjednodušují 
% % implementaci a~mírně urychlují asemblaci. 
% 
% 
% 
% 
% \section{Numerical methods}
% We implemented two numerical methods using the Deal II library (does not provide any XFEM functionality itself). 
% Implementing SGFEM method is still in progress. Lets discuss now in more details the XFEM implementation. We were inspired by
% work~\cite{gracie} of R.Gracie and J.R.Craig.
% 
% \subsection{Enrichment function choice}
% The hydro-geological wells represent sources with very small diameter in the model. If we solve a~local problem
% with circular domain with one well placed in the center, we will see the logarithmic dependence of the pressure head on the distance from the well.
% If we represented the well only by a~point, the pressure head would go to infinity while closing to the point (singularity $|\log 0|\rightarrow \infty$). 
% 
% To capture large gradients of pressure head around the wells, we introduce local enrichment function
% \begin{equation}
% \label{eqn:xshape_func}
% \phi_w(\mathbf{x}) = 
%   \begin{cases}
%   \log(r_w(\mathbf{x})), & r_w > R_w\\
%   \log(R_w), & r_w \le R_w,\\
%   \end{cases}
% \end{equation}
% where $r_w$ is the distance from the well center and $R_w$ is the well radius.
% 
% % Při modelování různých jevů pomocí metody konečných prvků (nejčastěji lineárních) se setkáváme s~problémy aproximace hledané veličiny v~okolí míst, 
% % kde má tato veličina velké gradienty nebo se jeví v daném měřítku nespojitě. 
% % Příkladem takového problému, kterým se budeme zabývat my, je úloha podzemního proudění ve vícezvodňovém systému, do kterého zasahují hydro\-geologické vrty. 
% % Vrty představují bodové zdroje (mají zanedbatelný rozměr vůči celé oblasti), v jejichž okolí má průběh tlaku logaritmický charakter, a~řešení modelu se 
% % v~bodě zdroje blíží nekonečnu (singularita $|\log 0|\rightarrow \infty$). Formulace modelu je inspirována článkem~\cite{gracie}.
% 
% 
% %Příkladem takových problémů mohou být zdroje v~polích teplotních, elekrostatických nebo v polích tlaku, které aproximujeme jako bodové. 
% % Průběh veličin popisujících tato pole má v~okolí těchto bodů logaritmický charakter a~řešení modelu se v~bodě zdroje blíží nekonečnu 
% % (singularita $|\log 0|\rightarrow -\infty$). Námi řešená úloha hledá aproximaci tlaku v~proudění podzemní vody v systému zvodní, do kterých zasahují vrty, 
% % představující zdroje. Formulace modelu je inspirována článkem~\cite{gracie}.
% 
% 
% \subsection{Discretization}
% We are using the so called corrected (modified) version of XFEM with ramp function $g_w$, introduced by T.P.Fries in~\cite{mxfem}. 
% We are looking for the pressure approximation in the form
% \begin{equation}
% \label{eqn:xfem_head}
% h(\mathbf{x})=\underbrace{\sum \limits_{j\in\mathcal{N}}\alpha_j \, \varphi_j(\mathbf{x})}_\textrm{FEM} + 
% \underbrace{\sum \limits_{w\in\mathcal{W}} \sum \limits_{k\in\mathcal{N}_w} g_w(\mathbf{x}) \beta_{w,k} \, \phi_w(\mathbf{x}) \varphi_k(\mathbf{x})}_\textrm{enrichment},
% \end{equation}
% where the first part with degrees of freedom $\alpha_j$ corresponds with the linear finite elements $\varphi_j$. The second part
% with degrees of freedom $\beta_{w,k}$ is the enrichment part. The index set $\mathcal{N}$ contains numbers of all mesh nodes, the index set $\mathcal{W}$ 
% contains numbers of all wells and $\mathcal{N}_w$ is the set of all enriched nodes from the well $w$.
% 
% We are enriching nodes of the mesh that are inside a circular neighborhood of given radius $r_{enr}$. The bigger the enriched area is, the larger amount of 
% additional degrees of freedom are coming from the enrichment. The proper choice of the radius $r_{enr}$ is not trivial and we would like to further 
% investigate this.
% 
% % Při řešení modelu metodou XFEM (přesněji její modifikovanou verzí s funkcí rampy $g_w$, více v~\cite{mxfem}) hledáme aproximaci tlaku v jedné zvodni ve tvaru
% 
% % kde první část, se stupni volnosti $\alpha_j$, odpovídá aproximaci pomocí lineárních konečných prvků $\varphi_j$ a druhá část, 
% % se stupni volnosti $\beta_{w,k}$, obohacení pomocí obohacující funkce $\phi_w$. Funkce volíme $\phi_w = \log(r)$ pro $r > R_w$ a $\phi_w =\log(R_w)$ 
% % pro $r \leq R_w$, kde $R_w$ je poloměr vrtu a~$r$~vzdálenost od středu vrtu. Množina $\mathcal{N}$ obsahuje indexy všech uzlů sítě, množina $\mathcal{W}$ 
% % indexy vrtů a množina $\mathcal{N}_w$ indexy obohacených uzlů od vrtu $w$. Obohacujeme uzly sítě, které spadají do oblasti v okolí vrtu dané zvoleným poloměrem.
% 
% 
% 
% 
% \section{Results}
% We implemented successfully the FEM method with adaptively refined mesh for the described model. Next we implemented
% the corrected XFEM method with linear finite elements and enrichment by logarithmic function.
% We showed that the XFEM method approximates the pressure head in the vicinity of the wells quite well and converges with order~$O(h^{1.7})$.
% Our results are in close agreement with the results from~\cite{gracie} where they show $O(h^{0.4})$ convergence for FEM and $O(h^{1.8})$ for XFEM.
% On the problem with more wells we also demonstrated large increase of memory usage with adaptive FEM method in comparison to XFEM.
% 

% Provedli jsme implementaci metody FEM s~adaptivně zjemňovanou sítí pro řešení popsaného modelu s~jednou zvodní. 
% Podařilo se implementovat modifikovanou metodu XFEM s~li\-neár\-ní\-mi konečnými prvky a~obohacením logaritmickou funkcí. 
% Na úloze, ve které známe analytické řešení, jsme ukázali, že metoda XFEM velmi dobře aproximuje tlakovou výšku v~okolí 
% vrtu a~rychle konverguje s řádem~$O(h^{1.7})$. Výsledky konvergence obou metod jsou srovnatelné s~článkem~\cite{gracie}, 
% kde uvádí pro FEM $O(h^{0.4})$ a~XFEM $O(h^{1.8})$. Na úlohách s více vrty jsme demonstrovali velký nárůst paměťové 
% a časové náročnosti výpočtu metodou FEM při zjemňování sítě oproti XFEM.

% \begin{figure}[!htb]
%   \vspace{0pt}
%       \begin{center}    
%         \includegraphics[width=0.6\textwidth]{\figpath convergence_graph.pdf}
%       \end{center}
%   \vspace{-10pt}
%   \caption[Konvergence metod FEM a XFEM]{Graf konvergence metod FEM (modrá) a~XFEM (červená). Body jsme proložili mocninnou funkcí, přičemž první dvě největší chyby řešení pomocí FEM jsme vynechali. Můžeme odhadnout řád konvergence metody FEM $O(h^{0.8})$ a~metody XFEM $O(h^{1.7})$.}
%   \vspace{-5pt}
%   \label{fig:convergence}
% \end{figure}

% \section{Conclusion}
% In the theoretical part of our work we derived the weak formulation from the equations (\ref{eqn:poisson}) and (\ref{eqn:well_eqs_test}) and we proved
% the existence and the uniqueness of the weak solution according to Lax-Milgram lemma (e.g. in~\cite{rek1}).
% 
% We created our own implementation of XFEM method using Deal II library and correlated our results with article~\cite{gracie}. 
% We showed its advantages and disadvantage of the method in comparison to adaptive linear FEM. 
% 
% \section{Future work}
% Currently we are working on implementing SGFEM method for our model and deeper investigation of the numerical properties of the linear system.
% 
% There are several aspects we would like to take interest in:
% \begin{itemize}
%   \item the choice of the size of the enrichment area (automatically if possible)
%   \item adaptive integration on the enriched elements -- there is an idea of precomputed quadrature points which would be then only mapped from reference element
%   \item improvement in solving the linear system
% \end{itemize}
% 
% After finishing this primal research of XFEM/SGFEM method we will aim our effort in implementation of XFEM in mixed hybrid method to compute
% both pressure head and velocity of water. We are interested also in theoretical aspect and we will try to prove existence and uniqueness of the solution
% and convergence of the method. Long term aim is to implement this method in software Flow123d which specializes in computations on complex meshes consisting 
% of simplicial elements of different dimensions.

% V teoretické části jsme odvodili slabou formulaci z~rovnic (1) a~(2) a~dokázáli existenci a~jednoznačnost slabého řešení pomocí Laxovy-Milgramovy věty (viz ~\cite{rek1}). 
% Navrhli jsme vlastní implementaci metody XFEM, ukázali její použití a přednosti, které nabízí oproti FEM, a~potvrdili výsledky uvedené v~článku~\cite{gracie}. 
% Navíc jsme vyzkoušeli pozměněnou formulaci modelu, ve které nahlížíme na vrty jako na plošné zdroje v~oblasti zvodní, což přineslo výše zmíněné výhody.
% Dále jsme v práci analyzovali blíže chybu obou metod a~zabývali se volbou velikosti obohacené oblasti v~metodě XFEM.

% Metoda XFEM nabízí mnoho dalších možností jak model dále rozvíjet a~optimalizovat -- volba velikosti oblasti obohacení, 
% adaptivní integrace na obohacených elementech, tvar členu obohacení a další. Bodové zdroje nejsou jediným problémem vhodným 
% pro řešení metodou XFEM, dalším námětem může být aproximace pole rychlostí pomocí XFEM ve smíšené formulaci a~aplikace v~puklinovém proudění.

% Podstatnou výhodou XFEM je jednoduchá tvorba sítě – není nutné, abychom elementy zjemňovali až na úroveň vrtů a~aby jejich hranice 
% aproximovala dobře hranici vrtu. Naopak výsledky FEM metody jsou v~tomto smyslu velice závislé na síti a~při reálné úloze není možné 
% takové požadavky splnit, neboť by značně rostla paměťová a~časová náročnost výpočtu.
% 
% \vspace{1cm}
% \textbf{Acknowledgement:}
% This work is made under the sincere guidance of Mgr. Jan B\v rezina, Ph.D.


\bibliographystyle{abbrv}
\bibliography{../citace}
     
     
     
\newpage

\newpage

%% Example of text sectioning
% \section{Introductionx}
% 
% The proceedings will contain the extended abstracts of lectures
% presented during the \textsc{Seminar on Numerical Analysis} (SNA).
% The papers in the range of 2 - 4 pages should be written in Czech or
% English and prepared in the \LaTeX2e system with the aid of the
% \texttt{sna} document style, see Fig.\,\ref{labfig}. Of course, the
% authors can use any of standard packages, e.g. \texttt{epsf} or
% \texttt{amsmath}.
% 
% %% Example of including a figure
% \begin{figure}[h]
%  \begin{center}
%   \begin{minipage}{50mm}
%    \begin{verbatim}
%     \documentclass{sna}
%     \begin{document}
%        Text ...
%     \end{document}
%    \end{verbatim}
%   \end{minipage}
%   \vspace{-8mm}
%  \end{center}
%  \caption{An example of an extended abstract source code.}
%  \label{labfig}
% \end{figure}
% 
% 
% \section{Text modifications}
% 
% The \texttt{imet} document class arose from the \texttt{article}
% style. Therefore, most of its original instruments can be
% exploited to modify the text of paper:
% \begin{itemize}
%  \item For \textit{the text sectioning}, the authors can use
%        standard commands \texttt{$\mathtt{\backslash}$section},
%        \texttt{$\mathtt{\backslash}$subsection}, etc. or their
%        equivalent unnumbered forms \texttt{$\mathtt{\backslash}$section*},
%        \texttt{$\mathtt{\backslash}$subsection*}, etc.
%  \item Also \textit{mathematical formulae} can be written
%        according to common standards as well as their references,
%        e.g.\,(\ref{linsys}).
% 
%        \begin{equation}
%         \label{linsys}
%         A u = f \, , \qquad u, f \in R^n
%        \end{equation}
% 
%  \item The authors can include \textit{figures and tables} into their
%        texts, e.g. see Fig.\,\ref{labfig} and Tab.\,\ref{labtab}. The
%        figures should be stored into EPS files and included by tools
%        of \texttt{epsf} package into the text. The descriptions should
%        be located bellow the figures and above the tables.
% 
%        \begin{table}[h]
%         \caption{An example of a table.}
%         \label{labtab}
%         \begin{center}
%          \begin{tabular}{|c|c|}
%           \hline
%           \textbf{Computer} & \textbf{Memory} \\
%           \hline
%            SUN HPC & shared      \\
%            IBM SP  & distributed \\
%           \hline
%          \end{tabular}
%         \end{center}
%        \end{table}
% 
%  \item It is possible to use \textit{references} to the bibliography,
%        e.g. \cite{t03, fh04}. The form of the bibliography is presented
%        at the end of this paper.
% \end{itemize}
% 
% \section{Conclusion}
% 
% After several or more concluding remarks, the authors can express their
% acknowledgement to institutions supporting their research.
% 
% \textbf{Acknowledgement:} This work has been supported by the grant XYZ-123.
% 
% \begin{thebibliography}{m}
% 
%  \bibitem{t03}
%   C. Third:
%   {\em My first article}.
%   In:
%   D. Fourth (ed.):
%   Proceedings of the conference ABC'03,
%   VSB-Technical university,
%   Ostrava,
%   2003,
%   pp. 100--120.
% 
%  \bibitem{fh04}
%   E. Fe, G. Hey:
%   {\em The second book}.
%   Cambridge University Press,
%   2004.
% 
% \end{thebibliography}

\end{document}
